\documentclass{beamer}
\usepackage[utf8]{inputenc}
\usepackage{lmodern}
\usepackage[brazil]{babel}
\usetheme{Madrid}

% -------------------------------------------------
% Identidade Visual - Centro Paula Souza
% -------------------------------------------------
\definecolor{cpsred}{RGB}{153,0,0}      % Vermelho institucional
\definecolor{cpsgray}{RGB}{85,85,85}    % Cinza técnico

\usecolortheme[named=cpsred]{structure}
\setbeamercolor{title}{fg=white,bg=cpsred}
\setbeamercolor{frametitle}{fg=white,bg=cpsred}
\setbeamercolor{structure}{fg=cpsred}
\setbeamercolor{normal text}{fg=black,bg=white}
\setbeamercolor{itemize item}{fg=cpsred}

\usepackage{ragged2e}
\usepackage{hyperref}
\usepackage{graphicx}

\setbeamertemplate{headline}{}
\setbeamertemplate{footline}[frame number]
\addtobeamertemplate{footline}{
  \hfill\usebeamercolor[fg]{frametitle}{\hspace{-1.5cm}\scriptsize Prof. MSc. Frederico Barbosa Muniz}\hspace{1cm}
}{}

% -------------------------------------------------
% Informações principais
% -------------------------------------------------
\title[MiteScan - DSM3]{\textbf{MiteScan - DSM3}}
\author{Desenvolvido por alunos da FATEC Registro}
\institute{Centro Paula Souza \\ FATEC Registro}
\date{}

% -------------------------------------------------
% Documento
% -------------------------------------------------
\begin{document}

% Capa
\begin{frame}
    \centering
    \vspace{1cm}
    {\color{cpsred}\Huge\textbf{MiteScan - DSM4}}\\[0.8cm]
    {\Large Projeto de Desenvolvimento de Sistemas}\\[0.4cm]
    \textcolor{cpsgray}{Centro Paula Souza}\\[0.2cm]
    \textcolor{cpsgray}{FATEC Registro}
\end{frame}

% Sumário
\begin{frame}{Sumário}
\tableofcontents
\end{frame}

% -------------------------------------------------
% SEÇÕES
% -------------------------------------------------

\section{Pitch}
\begin{frame}{Pitch}
\justifying
O \textbf{MiteScan} é um sistema inteligente para monitoramento de colmeias, que utiliza técnicas de \textbf{Aprendizado de Máquina} e \textbf{Processamento de Imagens} para detectar precocemente a presença do ácaro \textit{Varroa destructor}.  
\\[0.3cm]
O projeto visa apoiar o apicultor na gestão de suas colmeias, permitindo um controle mais eficiente, sustentável e automatizado das infestações.
\end{frame}

\section{Problematização}
\begin{frame}{Problematização}
\justifying
O ácaro \textit{Varroa destructor} é uma das principais ameaças às colônias de abelhas, causando enfraquecimento e morte das colmeias.  
\\[0.3cm]
A inspeção manual é demorada, sujeita a erros e depende da experiência do apicultor.  
\\[0.3cm]
O \textbf{MiteScan} busca automatizar a detecção do parasita, reduzindo perdas e melhorando o manejo apícola.  
\\[0.3cm]
\textbf{Fontes de imagem:}  
\scriptsize
\url{https://entomologytoday.org/.../honey-bee-with-varroa-mite-3/} \\
\url{https://pixabay.com/photos/honeybee-bee-varroa-parasite-mite-7339727/} \\
\url{https://www.dadant.com/learn/how-to-treat-varroa-mites-in-bees/}
\end{frame}

\section{Estado da Arte}
\begin{frame}{Estado da Arte}
\justifying
\textbf{Artigo 1:} \textit{Deep learning beehive monitoring system for early detection of the varroa mite}  
\\[0.2cm]
\textbf{Artigo 2:} \textit{Desarrollo de un sistema de reconocimiento y conteo del parasito Varroa destructor en colmenas para probar efectividad de tratamientos}  
\\[0.2cm]
\textbf{Artigo 3:} \textit{ApIsoT: An IoT Function Aggregation Mechanism for Detecting Varroa Infestation in Apis mellifera Species}  
\\[0.3cm]
Esses trabalhos inspiram a proposta do MiteScan, que agrega monitoramento ambiental, análise de imagem e integração com IoT.
\end{frame}

\section{Objetivos}
\begin{frame}{Objetivos}
\justifying
\textbf{Objetivo Geral:}  
Desenvolver um sistema inteligente de monitoramento apícola para detecção e prevenção de infestações por \textit{Varroa destructor}.  
\\[0.4cm]
\textbf{Objetivos Específicos:}
\begin{itemize}
    \item Implementar algoritmo de Aprendizado de Máquina e Processamento de Imagens;
    \item Identificar precocemente a presença do ácaro na colônia;
    \item Facilitar o diagnóstico e alerta ao apicultor;
    \item Configurar Arduino/ESP8266 para monitorar temperatura e umidade;
    \item Emitir alertas automáticos ao detectar anomalias.
\end{itemize}
\end{frame}

\begin{frame}{Objetivos Complementares}
\justifying
\begin{itemize}
    \item Monitorar variáveis ambientais que favorecem a proliferação do ácaro;
    \item Permitir o cadastro de colmeias e apicultores;
    \item Exibir dados de temperatura e umidade via aplicação web;
    \item Integrar com banco de dados e API para comunicação em tempo real.
\end{itemize}
\end{frame}

\section{Metodologia}
\begin{frame}{Metodologia (Fluxo 1)}
\justifying
\textbf{Fluxo de interação principal:}
\begin{itemize}
    \item O usuário realiza uma ação na aplicação web;
    \item Uma requisição HTTP é enviada para a API (FastAPI);
    \item A API processa a solicitação e consulta o banco de dados (AWS);
    \item Os dados retornam para a interface Front-End, exibindo as informações ao apicultor.
\end{itemize}
\end{frame}

\begin{frame}{Metodologia (Fluxo 2)}
\justifying
\textbf{Fluxo de detecção da Varroa:}
\begin{itemize}
    \item O microcontrolador ESP8266 realiza leitura de temperatura e umidade;
    \item Envia dados via protocolo MQTT para a API;
    \item A API encaminha as informações ao modelo de IA;
    \item O modelo processa a imagem e detecta o ácaro Varroa;
    \item O resultado é enviado ao Front-End e apresentado ao usuário.
\end{itemize}
\end{frame}

\section{Apresentação do Sistema}
\begin{frame}{Apresentação do Sistema}
\justifying
O sistema \textbf{MiteScan} integra hardware e software para o controle automatizado de colmeias:  
\\[0.3cm]
\begin{itemize}
    \item Interface Web para visualização e cadastro de colmeias;
    \item Módulo IoT com ESP8266 para leitura de dados ambientais;
    \item Modelo de IA para detecção de ácaros em imagens;
    \item Comunicação em tempo real via API e banco de dados na AWS.
\end{itemize}
\end{frame}

\section{Conclusão}
\begin{frame}{Conclusão}
\justifying
O projeto MiteScan contribui para o \textbf{Objetivo de Desenvolvimento Sustentável (ODS) 15}, relacionado à \textbf{vida terrestre}.  
\\[0.3cm]
O sistema se mostra mais eficiente que a inspeção visual tradicional, oferecendo \textbf{respostas rápidas} e \textbf{manejo adequado} das colmeias.  
\\[0.3cm]
Permite avaliar se a colmeia está suscetível a infestações e auxilia o apicultor em decisões de prevenção e tratamento.
\end{frame}

\end{document}
